% !TEX TS-program = pdflatex
% !TEX encoding = UTF-8 Unicode

\documentclass[12pt]{article}
\usepackage[utf8]{inputenc} 
\usepackage[slovene]{babel}
\usepackage{lmodern}
\usepackage{geometry}
\geometry{a4paper} 
\usepackage{graphicx}

%%% PACKAGES
\usepackage{booktabs} % for much better looking tables
\usepackage{array} % for better arrays (eg matrices) in maths
\usepackage{paralist} % very flexible & customisable lists (eg. enumerate/itemize, etc.)
\usepackage{verbatim} % adds environment for commenting out blocks of text & for better verbatim
\usepackage{subfig} % make it possible to include more than one captioned figure/table in a single float
\usepackage{amssymb}
\usepackage{amsmath}
\usepackage{amsthm}
\usepackage{amsfonts}
\usepackage{latexsym}
\usepackage{mathtools}
\usepackage{bbm}
\usepackage{float}

\newtheorem{definition}{Definicija}
\newtheorem{example}{Primer}
\newtheorem{theorem}{Izrek}
\newtheorem{lem}{Lema}
\newtheorem{prop}{Trditev}
\newtheorem{cor}{Posledica}
\newtheorem{remark}{Opomba}

\title{Magistrsko dela}
\author{Jaša Štefan}


\graphicspath{{C:/Users/Jasa/Desktop/faks/magistrska/magistrska-naloga/slike/}}



\begin{document}
\maketitle

\section{CLI}
\begin{theorem}{centralni limitni izrek - osnovna (klasična) verzija}

Naj bo $\{X_1, X_2, \dots, X_n\}$ zaporedje $n$ neodvisnih in enako porazdeljenih slučajnih spremenljivk z matematičnim upanjem $\mu$ in varianco $\sigma$. Naj bo slučajna spremenljivka $Z$ definirana kot $Z := \frac{\bar{X} - \mu}{\sigma / \sqrt{n}}$.  Potem je v limiti, ko $n \rightarrow \infty$, slučajna spremenljivka $Z$ porazdeljena standardno normalno.
\end{theorem}

Izrek nam pove, da tudi če delamo z drugače (nenormalno) porazdeljenimi slučajnimi spremenljivkami, njihova standardizirana vsota vseeno konvergira k standardno normalni porazdelitvi. Zaradi tega lahko za statistično modeliranje te vsote uporabimo orodja in metode, katere uporabljamo že pri standardno normalnih porazdelitvah. 


\section{Kolmogorov - Smirnova razdalja}




\section{Berry - Esseenov izrek}
Berry - Esseenov izrek oziroma neenakost nam pove, kako hitro porazdelitev standardiziranega povprečja konvergira proti normalni porazdelitvi, s tem ko omeji največjo napako aproksimacije med obema porazdelitvama. Točnost aproksimacije se meri z Kolmogorov-Smirnovo razdaljo. 

V primeru, da gre za neodvisne vzorce (slučajne spremenljivke), je hitrost konvergence $n^{-\frac{1}{2}}$. Konstanto ocenimo s pomočjo koeficienta simetrije. Trenutno je zgornja meja za konstanto 0,4748, spodnja meja pa 0,40973.

Izrek je zelo močan tudi zato, ker potrebujemo samo prve tri centralne momente.

\begin{theorem}{Berry - Esseenov izrek}
Naj bodo $X_1, X_2, \dots ,X_n$ neodvisne in enako porazdeljene slučajne spremenljivke z $E[X_1] = 0, E[X_1^2] = \sigma^2$ in $E[X_1^3] = \rho$, pri čemer $\sigma > 0$ in $\rho < \infty$.  Definirajmo slučajno spremenljivko $Y_n$, ki je vzorčno povprečje, $Y_n = \frac{X_1 + X_2+ \dots +X_n}{n}$. 
Nadalje definirajmo komulativno porazdelitveno funkcijo $F_n$ slučajne spremenljivke $\frac{Y_n}{\sigma\sqrt{n}}$, s $\Phi$ pa označimo komulativno porazdelitveno funkcijo standardno normalne porazdelitve. 

Potem obstaja konstanta $C, C>0$, da velja 
$$ |F_n(x) - \Phi(x)| \leq \frac{C \rho}{\sigma^3\sqrt{n}}.$$
\end{theorem}


\begin{example}{RULETA}
Igramo igro, kjer so možne vrednosti $0,1,\dots, 36$ in je verjetnost vsake izmed njih enaka. Vsakič napovemo 1 cifro. V primeru, da zadenemo, dobimo 35 enot, sicer 1 enoto izgubimo. Igro ponovimo tisočkrat, rezultati so med seboj neodvisni. 

Naj $X_1, X_2, \dots , X_3$  označujejo dobitek v vsaki igri.

Izračunamo $E[X_1] = - \frac{1}{37}, var(X_1) = 34,080$ in $skew(X_1) = 1162,366$.

Definiramo  $Y_n = \frac{X_1 +X_2 +\dots +X_{1000} - n\mu}{\sigma \sqrt{n}}$, $F_n$ pa naj bo definirana kot zgoraj.

Sledi $$|F_n(x) - \Phi(x)| \leq \frac{C \cdot 1162,366}{\sigma^3 \sqrt{n}}.$$

Za $C= 0.5$ dobimo rezultat  
$$|F_n(x) - \Phi(x)| \leq \frac{2.921}{\sqrt{n}}.$$

\end{example}

\begin{figure}[H]
    \centering
    \includegraphics[width=300px]{BE_ruleta.png}
\end{figure}

\begin{figure}[H]
    \centering
    \includegraphics[width=300px]{primerjava_z_normalno.png}
\end{figure}
\end{document}
